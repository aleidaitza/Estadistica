% Options for packages loaded elsewhere
\PassOptionsToPackage{unicode}{hyperref}
\PassOptionsToPackage{hyphens}{url}
%
\documentclass[
]{article}
\usepackage{amsmath,amssymb}
\usepackage{lmodern}
\usepackage{iftex}
\ifPDFTeX
  \usepackage[T1]{fontenc}
  \usepackage[utf8]{inputenc}
  \usepackage{textcomp} % provide euro and other symbols
\else % if luatex or xetex
  \usepackage{unicode-math}
  \defaultfontfeatures{Scale=MatchLowercase}
  \defaultfontfeatures[\rmfamily]{Ligatures=TeX,Scale=1}
\fi
% Use upquote if available, for straight quotes in verbatim environments
\IfFileExists{upquote.sty}{\usepackage{upquote}}{}
\IfFileExists{microtype.sty}{% use microtype if available
  \usepackage[]{microtype}
  \UseMicrotypeSet[protrusion]{basicmath} % disable protrusion for tt fonts
}{}
\makeatletter
\@ifundefined{KOMAClassName}{% if non-KOMA class
  \IfFileExists{parskip.sty}{%
    \usepackage{parskip}
  }{% else
    \setlength{\parindent}{0pt}
    \setlength{\parskip}{6pt plus 2pt minus 1pt}}
}{% if KOMA class
  \KOMAoptions{parskip=half}}
\makeatother
\usepackage{xcolor}
\IfFileExists{xurl.sty}{\usepackage{xurl}}{} % add URL line breaks if available
\IfFileExists{bookmark.sty}{\usepackage{bookmark}}{\usepackage{hyperref}}
\hypersetup{
  pdftitle={Tabla de frecuencias},
  pdfauthor={Aleida Jimenez Itza},
  hidelinks,
  pdfcreator={LaTeX via pandoc}}
\urlstyle{same} % disable monospaced font for URLs
\usepackage[margin=1in]{geometry}
\usepackage{color}
\usepackage{fancyvrb}
\newcommand{\VerbBar}{|}
\newcommand{\VERB}{\Verb[commandchars=\\\{\}]}
\DefineVerbatimEnvironment{Highlighting}{Verbatim}{commandchars=\\\{\}}
% Add ',fontsize=\small' for more characters per line
\usepackage{framed}
\definecolor{shadecolor}{RGB}{248,248,248}
\newenvironment{Shaded}{\begin{snugshade}}{\end{snugshade}}
\newcommand{\AlertTok}[1]{\textcolor[rgb]{0.94,0.16,0.16}{#1}}
\newcommand{\AnnotationTok}[1]{\textcolor[rgb]{0.56,0.35,0.01}{\textbf{\textit{#1}}}}
\newcommand{\AttributeTok}[1]{\textcolor[rgb]{0.77,0.63,0.00}{#1}}
\newcommand{\BaseNTok}[1]{\textcolor[rgb]{0.00,0.00,0.81}{#1}}
\newcommand{\BuiltInTok}[1]{#1}
\newcommand{\CharTok}[1]{\textcolor[rgb]{0.31,0.60,0.02}{#1}}
\newcommand{\CommentTok}[1]{\textcolor[rgb]{0.56,0.35,0.01}{\textit{#1}}}
\newcommand{\CommentVarTok}[1]{\textcolor[rgb]{0.56,0.35,0.01}{\textbf{\textit{#1}}}}
\newcommand{\ConstantTok}[1]{\textcolor[rgb]{0.00,0.00,0.00}{#1}}
\newcommand{\ControlFlowTok}[1]{\textcolor[rgb]{0.13,0.29,0.53}{\textbf{#1}}}
\newcommand{\DataTypeTok}[1]{\textcolor[rgb]{0.13,0.29,0.53}{#1}}
\newcommand{\DecValTok}[1]{\textcolor[rgb]{0.00,0.00,0.81}{#1}}
\newcommand{\DocumentationTok}[1]{\textcolor[rgb]{0.56,0.35,0.01}{\textbf{\textit{#1}}}}
\newcommand{\ErrorTok}[1]{\textcolor[rgb]{0.64,0.00,0.00}{\textbf{#1}}}
\newcommand{\ExtensionTok}[1]{#1}
\newcommand{\FloatTok}[1]{\textcolor[rgb]{0.00,0.00,0.81}{#1}}
\newcommand{\FunctionTok}[1]{\textcolor[rgb]{0.00,0.00,0.00}{#1}}
\newcommand{\ImportTok}[1]{#1}
\newcommand{\InformationTok}[1]{\textcolor[rgb]{0.56,0.35,0.01}{\textbf{\textit{#1}}}}
\newcommand{\KeywordTok}[1]{\textcolor[rgb]{0.13,0.29,0.53}{\textbf{#1}}}
\newcommand{\NormalTok}[1]{#1}
\newcommand{\OperatorTok}[1]{\textcolor[rgb]{0.81,0.36,0.00}{\textbf{#1}}}
\newcommand{\OtherTok}[1]{\textcolor[rgb]{0.56,0.35,0.01}{#1}}
\newcommand{\PreprocessorTok}[1]{\textcolor[rgb]{0.56,0.35,0.01}{\textit{#1}}}
\newcommand{\RegionMarkerTok}[1]{#1}
\newcommand{\SpecialCharTok}[1]{\textcolor[rgb]{0.00,0.00,0.00}{#1}}
\newcommand{\SpecialStringTok}[1]{\textcolor[rgb]{0.31,0.60,0.02}{#1}}
\newcommand{\StringTok}[1]{\textcolor[rgb]{0.31,0.60,0.02}{#1}}
\newcommand{\VariableTok}[1]{\textcolor[rgb]{0.00,0.00,0.00}{#1}}
\newcommand{\VerbatimStringTok}[1]{\textcolor[rgb]{0.31,0.60,0.02}{#1}}
\newcommand{\WarningTok}[1]{\textcolor[rgb]{0.56,0.35,0.01}{\textbf{\textit{#1}}}}
\usepackage{graphicx}
\makeatletter
\def\maxwidth{\ifdim\Gin@nat@width>\linewidth\linewidth\else\Gin@nat@width\fi}
\def\maxheight{\ifdim\Gin@nat@height>\textheight\textheight\else\Gin@nat@height\fi}
\makeatother
% Scale images if necessary, so that they will not overflow the page
% margins by default, and it is still possible to overwrite the defaults
% using explicit options in \includegraphics[width, height, ...]{}
\setkeys{Gin}{width=\maxwidth,height=\maxheight,keepaspectratio}
% Set default figure placement to htbp
\makeatletter
\def\fps@figure{htbp}
\makeatother
\setlength{\emergencystretch}{3em} % prevent overfull lines
\providecommand{\tightlist}{%
  \setlength{\itemsep}{0pt}\setlength{\parskip}{0pt}}
\setcounter{secnumdepth}{-\maxdimen} % remove section numbering
\ifLuaTeX
  \usepackage{selnolig}  % disable illegal ligatures
\fi

\title{Tabla de frecuencias}
\author{Aleida Jimenez Itza}
\date{2022-03-09}

\begin{document}
\maketitle

\hypertarget{importar-la-matriz-iris}{%
\section{1.- Importar la matriz iris}\label{importar-la-matriz-iris}}

\begin{Shaded}
\begin{Highlighting}[]
\FunctionTok{data}\NormalTok{(iris)}
\end{Highlighting}
\end{Shaded}

\hypertarget{exploracion-de-la-matriz-dimension-de-la-matriz-tiene-150-individuos-y-5-variables}{%
\section{Exploracion de la matriz dimension de la matriz tiene 150
individuos y 5
variables}\label{exploracion-de-la-matriz-dimension-de-la-matriz-tiene-150-individuos-y-5-variables}}

\begin{Shaded}
\begin{Highlighting}[]
\FunctionTok{dim}\NormalTok{(iris)}
\end{Highlighting}
\end{Shaded}

\begin{verbatim}
## [1] 150   5
\end{verbatim}

\hypertarget{nombre-de-las-variables}{%
\section{Nombre de las variables}\label{nombre-de-las-variables}}

\begin{Shaded}
\begin{Highlighting}[]
\FunctionTok{colnames}\NormalTok{(iris) }
\end{Highlighting}
\end{Shaded}

\begin{verbatim}
## [1] "Sepal.Length" "Sepal.Width"  "Petal.Length" "Petal.Width"  "Species"
\end{verbatim}

\hypertarget{tipos-de-variables}{%
\section{Tipos de variables}\label{tipos-de-variables}}

\begin{Shaded}
\begin{Highlighting}[]
\FunctionTok{str}\NormalTok{(iris) }
\end{Highlighting}
\end{Shaded}

\begin{verbatim}
## 'data.frame':    150 obs. of  5 variables:
##  $ Sepal.Length: num  5.1 4.9 4.7 4.6 5 5.4 4.6 5 4.4 4.9 ...
##  $ Sepal.Width : num  3.5 3 3.2 3.1 3.6 3.9 3.4 3.4 2.9 3.1 ...
##  $ Petal.Length: num  1.4 1.4 1.3 1.5 1.4 1.7 1.4 1.5 1.4 1.5 ...
##  $ Petal.Width : num  0.2 0.2 0.2 0.2 0.2 0.4 0.3 0.2 0.2 0.1 ...
##  $ Species     : Factor w/ 3 levels "setosa","versicolor",..: 1 1 1 1 1 1 1 1 1 1 ...
\end{verbatim}

\hypertarget{en-busca-de-factores-perdidos}{%
\section{En busca de factores
perdidos}\label{en-busca-de-factores-perdidos}}

\begin{Shaded}
\begin{Highlighting}[]
\FunctionTok{anyNA}\NormalTok{(iris) }
\end{Highlighting}
\end{Shaded}

\begin{verbatim}
## [1] FALSE
\end{verbatim}

\hypertarget{construccion-de-la-tabla-de-frecuencias}{%
\section{construccion de la tabla de
frecuencias}\label{construccion-de-la-tabla-de-frecuencias}}

Posicionarnos en una variable especifica Petal.Lenght indico que el
nombre me lo acorte a PL, lo que resulte de esa indicacion quiero que lo
que ponga en formato tabla, lo que resulte adquiera un formato de
data.frame A partir de lo anterior, voy a generar una nueva variable
(objeto) llamada \textbf{tabla\_PL}.

\begin{Shaded}
\begin{Highlighting}[]
\NormalTok{tabla\_PL}\OtherTok{\textless{}{-}}\FunctionTok{as.data.frame}\NormalTok{(}\FunctionTok{table}\NormalTok{(}\AttributeTok{PL=}\NormalTok{iris}\SpecialCharTok{$}\NormalTok{Petal.Length))}
\end{Highlighting}
\end{Shaded}

\begin{Shaded}
\begin{Highlighting}[]
\NormalTok{Petal\_Length}\OtherTok{\textless{}{-}}\FunctionTok{transform}\NormalTok{(tabla\_PL,}
          \AttributeTok{freqAC=}\FunctionTok{cumsum}\NormalTok{(Freq),}
          \AttributeTok{Rel=}\FunctionTok{round}\NormalTok{(}\FunctionTok{prop.table}\NormalTok{(Freq),}\DecValTok{3}\NormalTok{ ),}
          \AttributeTok{RelAC=}\FunctionTok{round}\NormalTok{(}\FunctionTok{cumsum}\NormalTok{(}\FunctionTok{prop.table}\NormalTok{(Freq)),}\DecValTok{3}\NormalTok{))}
\NormalTok{Petal\_Length}
\end{Highlighting}
\end{Shaded}

\begin{verbatim}
##     PL Freq freqAC   Rel RelAC
## 1    1    1      1 0.007 0.007
## 2  1.1    1      2 0.007 0.013
## 3  1.2    2      4 0.013 0.027
## 4  1.3    7     11 0.047 0.073
## 5  1.4   13     24 0.087 0.160
## 6  1.5   13     37 0.087 0.247
## 7  1.6    7     44 0.047 0.293
## 8  1.7    4     48 0.027 0.320
## 9  1.9    2     50 0.013 0.333
## 10   3    1     51 0.007 0.340
## 11 3.3    2     53 0.013 0.353
## 12 3.5    2     55 0.013 0.367
## 13 3.6    1     56 0.007 0.373
## 14 3.7    1     57 0.007 0.380
## 15 3.8    1     58 0.007 0.387
## 16 3.9    3     61 0.020 0.407
## 17   4    5     66 0.033 0.440
## 18 4.1    3     69 0.020 0.460
## 19 4.2    4     73 0.027 0.487
## 20 4.3    2     75 0.013 0.500
## 21 4.4    4     79 0.027 0.527
## 22 4.5    8     87 0.053 0.580
## 23 4.6    3     90 0.020 0.600
## 24 4.7    5     95 0.033 0.633
## 25 4.8    4     99 0.027 0.660
## 26 4.9    5    104 0.033 0.693
## 27   5    4    108 0.027 0.720
## 28 5.1    8    116 0.053 0.773
## 29 5.2    2    118 0.013 0.787
## 30 5.3    2    120 0.013 0.800
## 31 5.4    2    122 0.013 0.813
## 32 5.5    3    125 0.020 0.833
## 33 5.6    6    131 0.040 0.873
## 34 5.7    3    134 0.020 0.893
## 35 5.8    3    137 0.020 0.913
## 36 5.9    2    139 0.013 0.927
## 37   6    2    141 0.013 0.940
## 38 6.1    3    144 0.020 0.960
## 39 6.3    1    145 0.007 0.967
## 40 6.4    1    146 0.007 0.973
## 41 6.6    1    147 0.007 0.980
## 42 6.7    2    149 0.013 0.993
## 43 6.9    1    150 0.007 1.000
\end{verbatim}

\end{document}
